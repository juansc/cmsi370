\documentclass[11pt]{article}
\usepackage{geometry,tabularx,booktabs,float,natbib}                % See geometry.pdf to learn the layout options. There are lots.
\geometry{letterpaper}                   % ... or a4paper or a5paper or ... 
%\geometry{landscape}                % Activate for for rotated page geometry
%\usepackage[parfill]{parskip}    % Activate to begin paragraphs with an empty line rather than an indent
\usepackage{graphicx}
\usepackage{amssymb}
\usepackage{epstopdf}
\DeclareGraphicsRule{.tif}{png}{.png}{`convert #1 `dirname #1`/`basename #1 .tif`.png}
\newcommand{\myreferences}{references}

\title{Wearables}
\author{Juan S. Carrillo}
%\date{}                                           % Activate to display a given date or no date

\begin{document}
\maketitle
\section{Introduction}
One of the more interesting recent trends in the consumer market is the emergent interest in wearables. In fact, on KickStarter, a popular crowdfunding website, five of the ten top-funded projects are wearables \cite{VandricoTopTen}. All of the products managed to raise more that \$1 million from the general public. And it's not surprising. Some of these wearables seem to be science-fiction technology come true. The Emotiv Insight is a headset wearable which monitors and translates brain activity into signals that can be understood by other devices to remotely communicate with them, allowing the user to communicate with the devices through their thoughts. The Omate TrueSmart smartwatch can function as a phone, with capabilities such as allowing the users to browse the internet, make calls, and send text messages. Consumers are obviously intrigued by the possibilities of wearable computing, but realistically what can wearables really deliver to the consumer, given it's unique capabilities and limitations? This paper will focus on the physical capabilities of wearables, how they limit the functionality of wearables and how the practical functionality of a wearable differs from the expected functionality seen by users. We'll end this paper with a discussion on how their limitations and functionality can affect their role in the consumer market.
\section{Background}
 However, we wish to focus on the wearables that are available to consumers. There are wearables, such as the Brother AiRScouter, which are limited to industrial use. Wearables that are specific to a industry do not face the same usability issues that a wearable aimed at the general public face. For example, a person in industry might be \textit{required} to use a wearable for their job, and if necessary the individual's employer might provide training if the wearable device is difficult to learn. On the other hand, consumers \textit{choose} what wearable devices, based on perceived pragmatic(based on the functionality) and hedonic(based on the aesthetic and social appeal) qualities. Pragmatic qualities include compactness, comprehensibility, ease of use, portability, and interactivity, while hedonic qualities include aesthetic beauty, novelty, pleasure in use, and personalization\cite{UserAcceptance}. One study suggests that  style, price, convenience, and widespread assimilation play a huge role in whether or not will choose to buy a wearable \cite{WhyUsersDont}. These are issues that industrial use wearables do not generally face, and as such we will not consider industrial wearables in this paper. 

Before we analyze current wearables, we must first: 1) formally define wearables and 2) define a wearable \textit{should} do.  Wearables themselves are greatly varied in form and function. We must be careful to analyze the limitations that wearables face as a whole and not limit ourselves to a particular type of device such as a smartwatch or a headset. There are wearables for virtually any part of the body: there are headsets, googles, glasses, helmets, shirts, vests, bodysuits, armbands, watches, socks, and shoes \cite{VandricoList}. As such, we must define wearables to cover these and other forms that can arise in the future. Wearables, broadly, are ``body-worn devices, such as clothing and accessories, which integrate computational capabilities to provide specific features to users."\cite{WearableHumanView}. We note that this definition is can include devices such as regular digital wristwatches, step-counters, heart rate monitors, and even ankle-bracelets can be thought of, and are, wearables. However, they represent the most simple type of wearables, those that have a single type of functionality. Just as cellular phones began with simple phone calls and eventually gained more and more functionality, wearables have done the same. As wearables become more versatile and more functional it is of interest to study the limitations of what they can and cannot do. 

We will now consider what the \textit{ideal} wearable should do, regardless of form. Starner\cite{starnerChallenges1} provides a list of goals that, while ambitious, are a good standard to strive for. Wearables should:
\begin{enumerate}
    \item Persist and provide constant access to information services
    \item Sense and model context
    \item Adapt interaction modalities based on the user's context
    \item Augment and mediate interactions with the user's environment
\end{enumerate}

In order to \textit{persist and provide constant access to information services}, a wearable must be portable, capable of continuous use, and easily accessible. Starner adds that the device must be physically unobtrusive. A wearable must then be comfortable and discreet. This takes into account the dual nature of the wearable both as a computational device and a wearable accessory. Like a watch or a pair of glasses, a wearable must be immediately accessible yet entirely comfortable so that the user is not aware of the device until they need it.  This means that the device will most likely need to be small, lightweight, or otherwise invisible. The wearable can have the form of a familiar accessory such as a pair of glasses or a bracelet or even an article of clothing. The wearable should also be able to discretely interrupt in order to  
\section{Methods}
\section{Discussion}
\section{Conclusion}
\section{bibliography}
\bibliographystyle{unsrt}
\bibliography{\myreferences}

\end{document}  