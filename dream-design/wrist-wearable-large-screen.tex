\documentclass[11pt]{article}
\usepackage{geometry,tabularx,booktabs,float,natbib}                % See geometry.pdf to learn the layout options. There are lots.
\geometry{letterpaper}                   % ... or a4paper or a5paper or ... 
%\geometry{landscape}                % Activate for for rotated page geometry
%\usepackage[parfill]{parskip}    % Activate to begin paragraphs with an empty line rather than an indent
\usepackage{graphicx}
\usepackage{amssymb}
\usepackage{epstopdf}
\DeclareGraphicsRule{.tif}{png}{.png}{`convert #1 `dirname #1`/`basename #1 .tif`.png}
\newcommand{\myreferences}{references}

\title{Proposed Wrist-Mounted Smartdevice}
\author{Juan S. Carrillo}
%\date{}                                           % Activate to display a given date or no date

\begin{document}
\maketitle
\section{Introduction}
Wearables are growing in popularity. Recently, five out of the top ten crowdfunded campaigns on Kickstarter have been wearable devices. However, unlike the smartphone, it seems wearables have become more highly specialized, niche-specific products and fail to become standalone products in their own right. A large number of wrist-worn devices specialize as fitness trackers and do nothing more. More multipurpose products like Google Glass aren't completely independent; Google claims that in order for a ``great on-the-go Glass experience, it's essential to pair Glass to your phone or tablet." Overall, there are very few wearables that don't need to tether to a smartphone or other device in order to have full functionality. As a result it's very difficult for wearables to displace smartphones, which have much more functionality and just as much mobility as typical wearables. In this paper, I propose a device that combines the functionality of a smartphone with the mobility and wearability as a smartwatch.

\section{Background}
The functionality of wearables are greatly limited by their size and form. For example, head-mounted displays lack a graphical, direct manipulation oriented interface like the touch screen smartphones. Doing something as simple as sending a text message on Google Glass is significantly more difficult than on a smartphone because a user would have to resort to voice recognition software, which is highly error prone and requires the user to be in an environment with very little background noise. Smartwatches like the Pebble have a very small screens which limit the potential of incorporating touch screens into a smartphone. The small size of wearables limits their processing power as well as battery life. 

My design aims specifically to deal with issues and complaints common to smartwatches. Specifically, I aim to address the ``fat finger problem". The fat finger problem is that smartwatches have relatively small screens compared to the size of our fingers. Several approaches such as iterative zooming, buttons, gestures, skin buttons, knobs and movable watch heads in order to compensate for the small screen. However, I find these approaches unsatisfactory because they don't solve the fundamental problem of lacking space. Iterative zooming requires more taps and gestures to type in characters than a larger screen such as a cell phone or a tablet. Skin buttons make use of sensors that detect when a user a taps an icon illuminated on the users skin. Lighting conditions affect the effectivity of the sensors, and wearables should be equally effective regardless of the users setting. Buttons such as those included by the Pebble make it difficult for users to navigate a menu based system. Movable watch heads present moving parts which are often undesirable as they get worn over time. 

Of these approaches, the only one that attempts to increase the size of the interface is the skin buttons interface. I propose using a larger device with a larger, multitouch touchscreen. This solves the basic problems associated with tiny screens. My design borrows heavily from the design of smartphones, and I hope that the familiar design will appeal to users and allow usage of my device more intuitive and easy. Ultimately, the main advantage of a larger screen is that navigating the devices interface is much more easier. A larger device also allows for longer battery life and increased computational power. Admittedly, a larger device is also more obtrusive and more uncomfortable for users; I address these issues in another section of my paper.



\section{Physical Design}
I propose a wearable device that is worn on the left forearm. It would be roughly five to six inches in length and would wrap around the forearm entirely. It would be comprised of a rigid, curved, durable sapphire multitouch touchscreen that would curve around the top, inside, and lower side of the forearm as in the diagram below, held together by a piece of hinged piece of material that locks into place on the other end of the screen section. This would allow the user to put on the device by sliding the device on to the inner forearm and then securing the device by locking it in place. 

The device has two cameras, one right beneath the user's palm and the other in the middle of the solid plastic piece use for the latching mechanism. The camera below the user's palm would allow the user to have a front-facing camera when the user holds their arm with with the palm oriented towards the user's face. The second camera be a rear-facing camera when the user holds their arm such that the user's palm is perpendicular to the user's face. 

The screen is one, large continuous screen, although only one section of the screen will be active at any given time. With the use of gyroscopes and accelerometers, the device is able to detect its orientation and active the third of the screen that is most appropriate. I provide possible scenarios for the use of each third of the derives screen:

Here be the list

Now, I will talk about updates, mostly a tap(like iWatch) or vibrate thing. 

Now I will talk about why my device is awesome. It has a big screen, so now texting and surfing the internet will be a lot easier.

Talk about camera.

Talk about manual buttons.

\section{Design Rationale} 


\section{Advantages of Interface}

\section{Uses}
\section{Issues}
\section{User Response Predictions}
Possible drawbacks: discomfort, super expensive, battery life, device getting really hot. Not being discrete.

 
 A description of the type of system for which
you have created the design, focusing on any
particularly usability issues that you�d like to address
(see options in next section).
2. A top-level design or layout
3. At least two usage scenarios
4. Rationale for your design: relevant priorities,
mental models, interaction design concepts,
guidelines, principles, theories, etc.
5. Usability metric �forecast� analysis of your design�if
implemented then tested, what would 

\bibliographystyle{unsrt}
\bibliography{\myreferences}

\end{document}  